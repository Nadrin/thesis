\chapter{Theoretical framework for light simulation}

\section{Useful mathematical concepts}

\begin{df}[Spherical coordinates]
In spherical coordinate system position of a point $P$ in 3D space is described by values: $r$, $\theta$ and $\phi$, where:
\begin{itemize}
\item $r \geq 0$ (\emph{radial distance}) is the Euclidean distance from the origin to $P$.
\item $0 \leq \theta \leq \pi$ (\emph{polar angle}) is the angle between $OP$ line segment and a fixed axis ($OZ$).
\item $0 \leq \phi \leq 2\pi$ (\emph{azimuth angle}) is the angle between other axis ($OX$) and orthogonal projection of $OP$ on the reference plane ($XY$).
\end{itemize}
A point in spherical coordinates $(r, \theta, \phi)$ can be converted into cartesian coordinates $(x, y, z)$ as follows:
\begin{eqnarray}
  x &=& r \sin \theta \cos \phi \nonumber \\
  y &=& r \sin \theta \sin \phi \\
  z &=& r \cos \theta \nonumber
\end{eqnarray}
\end{df}

\begin{df}[Solid angle]
Solid angle subtended by an object visible from point $P$ is the area of that object projected onto a unit sphere around $P$. Solid angles are measured in \emph{steradians} $[sr]$. The entire sphere subtends a solid angle of $4\pi$.
\end{df}

In this work I will, by convention, denote a unit direction vector anchored at some point $P$ in space by $\omega$. Every such vector can be expressed by a point on a unit sphere in spherical coordinates:
\begin{equation}
  \omega(\theta,\phi) := \langle \sin\theta\cos\phi, \sin\theta\sin\phi, \cos\theta \rangle.
\end{equation}
Consider a differntial surface element $dS$ on a unit sphere spanning from $\theta$ to $\theta + d\theta$ and from $\phi$ to $\phi + d\phi$. Differential area of a set of directions $d\omega$ (or, equivalently, solid angle subtended by $dS$) is then
\begin{equation}
  d\omega = \frac{\sin\theta d\theta d\phi}{r^{2}} = \sin\theta d\theta d\phi.
\end{equation}
This relationship is the key to converting integrals over solid angles to integrals over directions.

\begin{df}[Projected solid angle]
Projected solid angle \parencite{nicodemus76} of an object $S$ measured from point $P$ is the solid angle of $S$ projected onto a unit disk perpendicular to the surface normal vector $\vec{N}(P)$ at $P$. Since $\vec{N}(P)$ defines the orientation of the projection plane the projected solid angle is cosine-weighted solid angle of $S$:
\begin{equation}
  \omega^{\perp} = |\cos\theta| \cdot \omega ,
\end{equation}
where $\theta$ is the angle between $\omega$ and surface normal at $P$. 
\end{df}

\section{Simulation assumptions}
\label{sec:assumptions}
Simulating light propagation in 3D scenes is very computationaly intensive task. It is generally reasonable to make certain assumptions about the nature of light or the scene itself that can drastically reduce algorithm complexity and speed up simulation without sacrificing noticable image quality. Some physical phenomena are insignificant at macroscopic scale or their effects are too miniscule relative to added simulation complexity.

The following properties are implicitly assumed for the rest of this work:
\begin{itemize}
\item Light transport is considered in the context of \emph{geometric optics} on macroscopic scale (quantumn effects are ignored).
\item Light travels in vaccum with no participating media and is unpolarized.
\item Light consists of radiation of mixed wavelengths and spectral effects, such as dispersion, are ignored.
\item The scene is in the state of energy equilibrium. 
\end{itemize}

\section{Radiometric quantities}
The natural basis for study of light transport in computer graphics is the field of \emph{radiometry} and, by lesser extent, \emph{photometry}.

Radiometry is the study of electromagnetic radiation (which, of course, incldues visible light spectrum) and its interaction with matter. Photometry, on the other hand, studies human perception of light. Defining key radiometric quantities is necessary for proper understanding of light transport theory.

Unless noted otherwise definitions in this section are based on chapter one of \cite{mccluney94}.

\begin{df}
Radiant power $[W=J \cdot s^{-1}]$ (or \emph{radiant flux}) is total amount of energy passing through a surface per unit time.
\begin{equation}
  \Phi := \frac{dQ}{dt}
\end{equation}
\end{df}

\begin{df}
Intensity $[W \cdot sr^{-1}]$ is radiant flux density per unit solid angle.
\begin{equation}
  I := \frac{d\Phi}{d\omega}
\end{equation}
It describes directional distribution of light and is only meaningful for point or small light sources.
\end{df}

\begin{df}
Irradiance $[W \cdot m^{-2}]$ is  area density of radiant flux arriving at a surface around point $x$.
\begin{equation}
  E(x) := \frac{d\Phi(x)}{dA(x)}
\end{equation}
Although formal definition of irradiance includes both sides of the surface in the measurement, for the purpose of computer graphics it is more convenient to consider only the side with normal vector $N(x)$ pointing outwards thus irradiance measures incident radiant flux in the upper hemisphere around point $x$.
\end{df}

\begin{df}
Radiant exitance $[W \cdot m^{-2}]$ \parencite{nicodemus78} is area density of radiant flux leaving a surface around point $x$.
\begin{equation}
  M(x) := \frac{d\Phi(x)}{dA(x)}
\end{equation}
Radiant exitance is analogous to irradiance but measures outgoing radiant flux around point $x$.
\end{df}

\begin{df}
Radiance $[W \cdot m^{-2} \cdot sr^{-1}]$ is radiant flux density per unit area per unit solid angle. It is measured on a differential surface perpendicular to $\omega$ and around point $x$.
\begin{equation}
  L(x,\omega) := \frac{d^{2}\Phi(x, \omega)}{dA(x) |\cos\theta| d\omega},
\end{equation}
where $\theta$ is the angle between $\omega$ and surface normal $\vec{N}(x)$. Radiance can also be formulated by using the dot product of $\omega$ and surface normal:
\begin{equation}
  L(x, \omega) := \frac{d^{2}\Phi(x, \omega)}{|\omega \cdot \vec{N}(x)| dA(x) d\omega}.
\end{equation}
\end{df}

It is often useful to distinguish between radiance arraving at a surface and radiance leaving a surface. Throughout this work \emph{incident radiance} will be denoted by $L_{i}(x, \omega)$ while \emph{exitant radiance} will be denoted by $L_{o}(x, \omega)$. Following from assumptions in section \ref{sec:assumptions} it is easly seen that:
\begin{equation}
  L_{i}(x, \omega) = L_{o}(x, -\omega).
\end{equation}
In the rest of this work 

\section{Reflection models}
The next step in formulating mathematical model for light transport is the introduction of the \emph{bidirectional reflectance distribution function} or BRDF for short. The formulation here is based on \cite{veach97}.

\subsection{The BRDF}
Let $x$ be a point on a surface with corresponding normal vector $N(x)$. Consider an outgoing direction $\omega_{o}$ and exitant radiance $L_{o}(x, \omega_{o})$. For a fixed incoming direction $\omega_{i}$ the irradiance due to incident light from an infinitesimal cone $d\omega_{i}$ around $\omega_{i}$ is
\begin{equation}
  dE(x, \omega_{i}) = L_{i}(x, \omega_{i}) |\cos\theta_{i}| d\omega_{i}.
\end{equation}
Recall from section \ref{sec:assumptions} that the modelling of light in this work assumes principles of geometric optics, that includes linearity of light. From this assumption it follows that exitant radiance at $x$ in direction $\omega_{o}$ is linearly proportional to the irradiance
\begin{equation}
  dL_{o}(x, \omega_{o}) \propto dE(x, \omega_{i}).
\end{equation}

\begin{df}[BRDF]
Bidirectional reflectance distribution function $[sr^{-1}]$ is the constant of proportionality of outgoing radiance due to irradiance at point $x$ on some surface.
\begin{equation}
  f_{r}(\omega_{o}, \omega_{i}) := \frac{dL_{o}(x, \omega_{o})}{dE(x, \omega_{i})} = \frac{dL_{o}(x, \omega_{o})}{L_{i}(x, \omega_{i}) |\cos\theta_{i}| d\omega_{i}}.
\end{equation}
\end{df}

In other words BRDF is radiance leaving in direction $\omega_{o}$ per unit of irradiance arriving from $\omega_{i}$. Alternative notation for the BRDF function is $f_{r}(\omega_{i} \rightarrow \omega_{o})$.

Physically based BRDFs have the following properties:
\begin{itemize}
\item \emph{Reciprocity}: For all pairs of directions $\omega_{i}$, $\omega_{o}$: $f_{r}(\omega_{o}, \omega_{i}) = f_{r}(\omega_{i}, \omega_{o})$.
\item \emph{Energy conservation}: The total energy of reflected light is less than or equal to the enrgy of incident light.
\end{itemize}

\begin{df}
Directional--hemispherical reflectance \parencite{sillion94} is the fraction of incident radiant flux density in a given direction that is reflected over the entire hemisphere of possible directions.
\begin{equation}
\label{eq:dhrefl}
  \rho_{dh}(\omega_{i}) = \int_{\Omega} f_{r}(\omega_{o}, \omega_{i}) |\cos\theta_{o}| d\omega_{o}
\end{equation}
\end{df}

\subsection{Lambertian reflection}
The simplest, yet widely used, reflection model is the \emph{Lambertian diffuse reflection}. Lambertian surfaces scatter incident light uniformly in all directions. While such ideal diffuse reflection is physically unplausable, in practice, it is fairly good approximation of many real life materials like matte plastic or clay.

Let $f_{d}$ be the BRDF of perfectly diffuse surface. Due to BRDF reciprocity $f_{d}(\omega_{o}, \omega_{i}) = f_{d}(\omega_{i}, \omega_{o})$, therefore $f_{d}$ is independent of both $\omega_{i}$ and $\omega_{o}$ and thus constant:
\begin{equation}
  f_{d}(\omega_{o}, \omega_{i}) = c.
\end{equation}
Substitution into equation (\ref{eq:dhrefl}) yields \emph{diffuse reflectance} coefficient:
\begin{eqnarray}
  \rho_{d} &=& c \int_{\Omega} |\cos\theta_{o}| d\omega_{o} \\
  &=& c \int_{0}^{\pi} \int_{0}^{2\pi} |\cos\theta_{o}| \sin\theta_{o} d\theta_{o} d\phi_{o} \\
  &=& c \pi.
\end{eqnarray}
In terms of diffuse reflectance the Lambertian BRDF is then
\begin{equation}
  f_{d}(\omega_{o}, \omega_{i}) = \frac{\rho_{d}}{\pi}.
\end{equation}

\subsection{Specular reflection}

\subsection{Fresnel reflectance}

\section{The light transport equation (LTE)}

