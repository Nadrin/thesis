\chapter{Introduction}
Synthesizing realistic images on computers has been an open research problem in computer science for many years now. Physically-based techniques of light simulation are the most robust methods of achieving appartent photo-realism in rendered images. Up until recently the computational cost of physical light simulation was prohibitive on commodity hardware and as a result the domain of realistic rendering was limited to big computational clusters and specialized graphics workstations.

With the emergence of fully programmable graphics processing units (GPUs) personal computers have been slowly but steadily gaining the needed power to compete with specialized hardware. Modern high-end graphics cards offer computational throughput in orders of teraflops per second and with their highly parallel nature are very well suited for algorithms like raytracing.

In the recent years there has been an ongoing effort to shift professional computational graphics from CPU-based solutions to high end GPUs. Although general purpose programming on GPUs has deemed more challenging to the programmers it often offers exceptional performance surpasing highly optimised CPU-centric algorithms by an order of magnitude.

In this thesis I present the implementation of \emph{Aurora Renderer}: a physically-based Monte Carlo distribution raytracer for modern GPUs. Aurora uses the NVIDIA CUDA technology to accelerate rendering on any GPU with CUDA compute capability 2.0 or higher. Aurora is fully integrated with Autodesk Maya -- an industry standard 3D content creation suite used by professionals in the film, VFX and game development industry.

The full source code, licensed under the MIT open source license, is available on the project page: \url{http://www.siejak.pl/projects/aurora}.
\vfill

\section{Structure of the thesis}
Below is a brief overview of each chapter of the thesis.

\subsection{Chapter \ref{ch:theory} -- Theoretical framework for light simulation}
This chapter introduces the theory behind computational light simulation. Basic mathematical tools are revised and crucial radiometric terms and quantities are defined and related to each other. The bidirectional reflectance distribution function is derived along with several reflection models. The chapter concludes with formulation of the light transport equation describing the energy equilibrium in a 3D scene.

\subsection{Chapter \ref{ch:montecarlo} -- Monte Carlo integration}
Here, a method of numerically solving the light transport equation introduced in the previous chapter is proposed. The basic Monte Carlo integration estimator is derived and the basic instruments for arbitrary distribution sampling are presented to the reader. Several essential 2D sampling strategies are derived and the concept of importance sampling is introduced. The chapter concludes with formulation of the Monte Carlo integration estimator using multiple importance sampling.

\subsection{Chapter \ref{ch:raytracing} -- Distribution ray tracing on the GPU}
This chapter describes the implementation details of Aurora renderer. An overview of high-level architecture is presented followed by introduction of the raytracing algorithm. An effictient ray/scene intersection test is devised using parallel geometry presorting for implicitly representing spatial intersection accelerator. Geometry format optimised for GPU rendering is briefly described and the implementation details of various types of lights and surface shaders are presented. A simple pinhole camera model is introduced. Monte Carlo algorithm for numerically estimating scene illumination and a sampling method for final image reconstruction serve as a conclusion for this chapter.

\subsection{Chapter \ref{ch:results} -- Conclusions and results}
The final chapter of the thesis summarizes rendering results achieved with Aurora. A number of output renders are presented and possible future improvements are discussed.



